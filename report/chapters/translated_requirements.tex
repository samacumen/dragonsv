%Lists all the translated global requirements
% This must cover all possible scenario for each of the global requirements that we decided.
% I have tried to list out all possiblities (in a way that modal-u-calculus) can be written for these translations (sounding negative is important to formulate modal formula much easily). Refer to the report from last year. Feel free to include anything that I may have missed out. (Leave a comment also so that we are aware what changed)

This chapter lists the requirements from chapter\ref{sec:global_req} in terms of interactions described in chapter\ref{sec:ext_interactions}.

\textit{Note: As the running of verification with larger number of ids in the mCRL2 tool takes considerable amount of time, these modal formulae are defined for restricted number of different package ids.
The number of racks in the system is defined by “MaxPositions” variable.
The number of different ids for packets is defined by “maxPacketID” variable.}\\

\begin{enumerate}
\item \textit{Each elevator, rack and conveyor belt contains at most one packet}

% Number 1.1 
	\textit{Conveyor Belt}
	\begin{itemize}
	\item
	Whenever an AcceptPacket(id: packetID) action is performed it is not possible to perform another AcceptPacket(id2: packetId) unless a ConvToElevator(e: elevID) action is performed in the meanwhile (where, e can be any one of the two elevators)
	
	\textbf{
	forall id, id2: PacketID . val(id $<$ maxPacketID \&\& id2 $<$ maxPacketID \&\& id $!=$ id2) $=>$ [true*. AcceptPacket(id). (!(ConvToElev(C3) $||$ ConvToElev(C4)))*. AcceptPacket(id2)] false}
	
Note that ConvToElev(e) action is extended to ``ConvToElev(C3) $||$ ConvToElev(C4)`` in order to prevent the scenario of finding a trace to ``ConvToElev(C4)`` when the tool is evaluating formula with ``ConvToElev(C3)``. In other words, we prevent all possible scenario of ConvToElev(e), for our two elevators. This puts a stronger restriction.
	
% Number 1.2	
	\item Whenever ElevToConv(e: elevID) action is performed it is
	not possible to perform another ElevToConv(e: 
	elevID) unless a DeliverPacket(id: packetID) action is performed in the meanwhile.
	
	\textbf{forall e: elevID. [true*. ElevToConv(e). (!(DeliverPacket(1) $||$ DeliverPacket(2)))*. ElevToConv(e)]false}

	
Note that DeliverPacket(id) action is extended to ``DeliverPacket(1) $||$ DeliverPacket(2)`` in order to prevent the scenario of finding a trace to ``DeliverPacket(2)`` when the tool is evaluating formula with ``DeliverPacket(1)``. Similar to the above case, we prevent all possible scenario of DeliverPacket(id), restricted to two $ids$.

\end{itemize}
	
% Number 1.3.1 & 1.3.2
	\textit{Elevator}
	\begin{itemize}
	\item Whenever a ConvToElev(e: elevID) or RackToElev(e: elevID) 
	action is performed it is not possible to perform another
	ConvToElev(e: elevID) or RackToElev(e: elevID) unless an 
	ElevToRack(e: elevID) or ElevToConv(e: elevID) action is performed.
	
	\textbf{forall e : elevID, p: Position .val(p $>$ -2 \&\& p $!=$ 0 \&\& p $<=$ MaxPositions) $=>$ [true*.(RackToElev(e,p)).(!(ElevToRack(e,p) $||$ ElevToConv(e)))*. (RackToElev(e,p))] false }
	
	\textbf{forall e: elevID, p: Position. val(p $>$ -2 \&\& p $!=$ 0  \&\& p $<=$ MaxPositions) $=>$ [true*. ConvToElev(e). (!(ElevToRack(e,p) $||$ ElevToConv(e)))* .ConvToElev(e)] false	}
	\end{itemize}

% Number 1.4
	\textit{Rack}
	\begin{itemize}
	\item Whenever an ElevToRack(e: elevID, p:Position) action is done it is not possible to perform another ElevToRack(e: elevID, p:Position) unless a RackToElev(e: elevID,p:Position) action is performed.
	
	\textbf{forall e: elevID, p : Position .val(p $>$ -2 \&\& p $!=$ 0 \&\& p $<=$ MaxPositions) $=>$ [true*. ElevToRack(e,p). !RackToElev(e,p)*. ElevToRack(e,p)] false }
	\end{itemize}

% Number 2
\item \textit{ Packet is exchanged only when the elevator platform is at
the same level as that of a conveyor belt}
	\begin{itemize}
	\item Whenever a ConvToElev(e: elevID) or a ElevToConv(e: elevID) is done, a MoveElevator(e :elevID, 0: Position) must be performed before, without any successive MoveElevator(e :elevID, p: Position) action in the meanwhile; where p is a position different than 0. (Assumption: Input and output conveyor are at position zero).
	
	\textbf{forall e:elevID, p : Position .val(p $>$ -2 \&\& p $!=$ 0 \&\& p $<$ MaxPositions) $=>$ [true*. MoveElevator(e,p). (ConvToElev(e) $||$ ElevToConv(e))] false}
	
	\end{itemize}

%Number 3
\item \textit{Packet is exchanged only when elevator platform is at the same level as that of a rack}
	\begin{itemize}
	\item ElevToRack(e: elevID, p: Position) and RackToElev(e: elevID, p:Position) action can only be performed if the last type of action of elevator controller is MoveElevator(e: elevID, p: Position).
	\end{itemize}
	
		\textbf{forall e: elevID, p : Position . val( p $>$ -2 \&\& p $<=$ MaxPositions) $=>$ [(!MoveElevator(e,p))*. ElevToRack(e,p)] false}

% NUmber 4
\item \textit{The lower elevator must never pass the upper one}
	\begin{itemize}
	\item Controller C3 cannot initiate the MoveElevator(C3: elevID, P3: Position) action if the last type of action of controller C4 is MoveElevator(C4: elevID, P4: Position).\\
	
	\textbf{forall P1, P2, P3, P4: Position. val((P1 $>$ -1 \&\& P1 $<=$ MaxPositions) \&\& (P2 $>=$ -1 \&\& P2 $<$ MaxPositions) \&\& (P3 $>=$ -1 \&\& P3 $<=$ MaxPositions) \&\& (P4 $>=$ -1 \&\& P4 $<$ MaxPositions) \&\& P3 $<$ P4 ) $=>$ $\\$ [ true*. MoveElevator(C4,P4). (!(MoveElevator(C3,P1) $||$ $\\$ MoveElevator(C4,P2))*). MoveElevator(C3,P3) ] false }

Controller C4 cannot initiate the MoveElevator(C4: elevID, P4: Position) action if the last type of action of controller C3 is MoveElevator(C3: elevID, P3: Position). 

	\textbf{forall P1, P2, P3, P4: Position. val((P1 $>$ $-1$ \&\& P1 $<=$ MaxPositions) \&\& (P2 $>=$ $-1$ \&\& P2 $<$ MaxPositions) \&\& (P3 $>=$ $-1$ \&\& P3 $<=$ MaxPositions) \&\& (P4 $>=$ $-1$ \&\& P4 $<$ MaxPositions) \&\& P3 $<$ P4 ) $=>$ $\\$ [ true*. MoveElevator(C3,P3). (!(MoveElevator(C3,P1) $||$ MoveElevator(C4,P2))*). MoveElevator(C4,P4) ]false }
	
\textit{Note: Here, we assume C4 controller to control the lower elevator and C3 controls the upper elevator. }

\end{itemize}
	
% Number 5
\item \textit{Packets are always delivered in the same order as
	requested}	
	\begin{itemize}
	\item 
	As long as the packet storage system has the ``id: PacketID`` and ``id2: PacketID``, for all WantOut(id) and a following WantOut(id2), without a DeliverPacket(id) in the meanwhile, it should be never be possible to perform DeliverPacket(id2) without DeliverPacket(id) before.
	
   \textbf{forall id, id2:PacketID. val(id $!=$ id2 \&\& id $<$ maxPacketID $\\$ \&\& id2 $<$ maxPacketID) $=>$ [(!(WantIn(id) $||$ WantIn(id2)))*. \\ AcceptPacket(id).(!WantOut(id))*. AcceptPacket(id2). (!WantOut(id2))*. WantOut(id). (!WantOut(id2))*. WantOut(id2). (!(DeliverPacket(id)))*. DeliverPacket(id2)] false }
   
 \textit{In the first part of formula accepting the packets in the system is described. After that ''WantOut'' actions are taken one by one and eventually delivering of the wrong packet part is implemented at the end of formula.}
   
\end{itemize}

% Number 6	
\item \textit{If a packet is ready to enter and there is a free
	position at the rack(s), it will be eventually accepted}
	\begin{itemize}
	\item Whenever the difference in the number of 'AcceptPacket' actions and number of 'DeliverPacket' actions is smaller than or equal than the ``number of racks + 1`` in the system, a new AcceptPacket(id: packetID) action should be possible to take.
	
	\textbf{forall id:PacketID. val(id $<$ maxPacketID) $=>$ $\\$ nu X(n: Int = $0$). val(n $>=$ 0 \&\& n $<=$ MaxPositions + 1) $=>$ $\\$ [(!WantIn(id))*. WantIn(id)] (mu Y. $<$!(AcceptPacket(id) $||$ RejectPacket(id))$>$ Y $||$  $<$AcceptPacket(id)$>$ X(n+1) $||$ $<$RejectPacket(id)$>$ X(n)) \&\& [!(AcceptPacket(id) $||$ DeliverPacket(id))] X(n) \&\& [DeliverPacket(id)]  X(n-1)}
	
In the formula above, two fixed point modalities are used in a nested fashion. The inner fixed point formula calculates the minimum set of states where we reach “AcceptPacket” or “RejectPacket” action after “WantIn” action. “WantIn” action is supposed to result in AcceptPacket action, as there being a vacancy in the system for packets must allow us to accept a new packet. However, the mechanism, which is implemented in order to prevent accepting different packets with same “id”, creates the traces which results in “RejectPacket” action. Therefore, in the inner fixed point modality both AcceptPacket and RejectPacket actions are handled. In addition to these, the outer fixed point modality is responsible for counting the number of packages in the system.
	
\end{itemize}

% Number 7		
\item \textit{If a requested packet is in the system, it will be
	eventually delivered}
	\begin{itemize}
	\item Whenever a WantOut(id: packetID) action is performed, after an AcceptPacket(id: packetID) then eventually a DeliverPacket(id: packetID) must be performed unless DeliverPacket(id: packetID) is performed before WantOut(id: packetID) action.
	
	\textbf{forall id: PacketID.val(id $<$ maxPacketID) $=>$ $\\$ [(!AcceptPacket(id))*. AcceptPacket(id). (!(DeliverPacket(id)))*.\\ WantOut(id)] $<$(!DeliverPacket(id))*.DeliverPacket(id)$>$ true}
	
\end{itemize}

% Number 8	
\item \textit{If a packet is unable to be located, the user requesting the packet is informed }
	\begin{itemize}	
	\item  Whenever a WantOut(id: packetID) action is performed after an 
	AcceptPacket(id: packetID) action, an PacketNotFound(id: packetID) action must be performed if a DeliverPacket(id: packetID) action is not performed.

This requirement requires to find a path where the packet is never delivered to the output after having accepted the packet into the system. Since, no such path must exist to have the requirement satisfied, and having no external interaction (expect for PacketNotFound action) to describe this process, this requirement was not verified as no straightforward way was found.

A simpler version is: [true*. PacketNotFound(id)]$<$true$>$true, which describes whenever PacketNotFound action is found, the scenario being described is true and PacketNotFound in itself informs the user of not having found the packet. However, this may not describe the requirement in its fullest terms.
 
\end{itemize}

% Number 9	
\item \textit{The number of packets in the system can at most be equal to the number of racks}
	\begin{itemize}
	\item •	The difference in the number of “AcceptPacket” actions and number of “DeliverPacket” actions can at most be equal to the number of racks in the system.
	
\textbf{forall id:PacketID. val( id<maxPacketID ) $=>$ \\ nu X(n:Int =$0$). [!(AcceptPacket(id) $||$DeliverPacket(id))] X(n)\\ \&\& [AcceptPacket(id)] X(n+1) \&\& [DeliverPacket(id)] X(n-1)\\ \&\& val(n $>=$ $0$ \&\& n $<=$ $MaxPositions+1$)}	

In this formula we introduce fixed point modality, to find the maximum number of states for the condition \textit{(n = 0 to MaxPositions+1)}. The result of the computation keeps the total packet bound within the maximum number of racks, thus fulfilling the requirement. Note that, we can have MaxPosition + 1 packets in the system.

	\end{itemize}
\end{enumerate}
%End of translations.