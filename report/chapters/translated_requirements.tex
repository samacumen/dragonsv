%Lists all the translated global requirements
% This must cover all possible scenario for each of the global requirements that we decided.
% I have tried to list out all possiblities (in a way that modal-u-calculus) can be written for these translations (sounding negative is important to formulate modal formula much easily). Refer to the report from last year. Feel free to include anything that I may have missed out. (Leave a comment also so that we are aware what changed)

This chapter lists the requirements from chapter\ref{sec:global_req} in terms of interactions described in chapter \ref{sec:ext_interactions}.

\begin{enumerate}
\item \textit{Each elevator, rack and conveyor belt contains at most one packet}

	\textit{Conveyor Belt}
	\begin{itemize}
	\item
	Whenever an AcceptPacket(id: packetID) action is performed it
	is not possible to perform another AcceptPacket(id:
	packetId) unless a ConvToElevator(e: elevID) action is
	performed in the meanwhile (where, e can be one of the two elevators)
	\item Whenever ElevToConvOut(e: elevID) action is performed it is
	not possible to perform another ElevToConvOut(e: 
	elevID) unless a DeliverPacket(id: packetID) action is performed in the meanwhile.
	\end{itemize}
	\textit{Elevator}
	\begin{itemize}
	\item Whenever a ConvToElev(e: elevID) or RackToElev(e: elevID) 
	action is performed it is not possible to perform another
	ConvToElev(e: elevID) or RackToElev(e: elevID) unless an 
	ElevToRack(e: elevID) or ElevToConv(e: elevID) action is performed.
	\end{itemize}
	
	\textit{Rack}
	\begin{itemize}
	\item Whenever an ElevToRack(p: pos, r: rackID) action is done it is not
	possible to perform another ElevToRack(p: pos, r: rackID) unless a RackToElev(p: pos, r: rackID) action is performed.%What about ConvToElev() missing?
	\end{itemize}

\item \textit{ Packet is exchanged only when the elevator platform is at
the same level as that of a conveyor belt}
	\begin{itemize}
	\item Whenever a ConvInToElev(e: elevID) or a ElevToConvOut(e: 
	elevID) is done a MoveElevator(e :elevId, 0: pos) must be
	performed before, without any successor MoveElevator(e :elevId, p: pos) action in the meanwhile where p is a position different than 0.\\(Assumption: Input and output conveyor are at
	position zero.).	
	\end{itemize}

\item \textit{Packet is exchanged only when elevator platform is at the same level as that of a rack}
	\begin{itemize}
	\item ElevToRack(e: elevID, r: rackID) and RackToElev(e: elevID, r: rackID) action can only be performed if the last type of action of elevator controller is MoveElevator(e: elevID, p: pos)..
	\end{itemize}
	
\item \textit{The two elevators cannot be at the same position}
	\begin{itemize}
	\item If there is a last action from type MoveElevator(e: elevID, p: pos),
	there cannot be a MoveElevator(e': elevID, p: pos) action unless 
	MoveElevator(e: elevID, p': pos) action takes place.\\(Assumption: where p and p' are different position.)

	\end{itemize}
		
\item \textit{The lower elevator must never pass the upper one}
	\begin{itemize}
	\item Controller C4 cannot initiate the MoveElevator(e: elevID, p: pos) 
	action if the last type of action of controller C3 is MoveElevator(e': elevID, p': pos).\\
	\textit{Note:} Here, p $>$ p' and e, e' is the lower and 
	upper elevator, respectively. We assume C4 controller to control
	the lower elevator and C3 controls the upper elevator.
	\end{itemize}
	
\item \textit{Packets are always delivered in the same order as
	requested}	
	\begin{itemize}
	\item 
	For all WantOut($id$: packetID) and a following WantOut($id'$: 
	packetID), \textit{without a DeliverPacket(id: packetID) in the meanwhile}, it
	should be never be possible to perform DeliverPacket($id'$: 
	packetID) followed by DeliverPacket($id$: packetID).
	\end{itemize}
	
\item \textit{If a packet is ready to enter and there is a free
	position at the rack(s), it will be eventually accepted}
	\begin{itemize}
	\item Whenever the difference in the number of 'AcceptPacket'
	actions and number of 'DeliverPacket' actions is  smaller than the
	number of racks in the system, a new AcceptPacket(id: packetID)
	action should be possible.
	\end{itemize}
	
\item \textit{If a requested packet is in the system, it will be
	eventually delivered}
	\begin{itemize}
	\item Whenever a WantOut(id: packetID) action is performed, after an AcceptPacket(id: 
	packetID) then eventually a DeliverPacket(id: packetID) must be
	performed unless DeliverPacket(id: packetID) is performed before WantOut(id: packetID) action. 
	\end{itemize}
	
\item \textit{If a packet is unable to be located, an alarm must 
	be generated}
	\begin{itemize}	
	\item  Whenever a WantOut(id: packetID) action is performed after an 
	AcceptPacket(id: packetID) action, an PacketNotFound(id: packetID) action must be performed if a 
	DeliverPacket(id: packetID) action is not performed.
	\end{itemize}
		
\item \textit{The number of packets in the system can at most be equal to the number of racks}
	\begin{itemize}
	\item The difference in the number of 'AcceptPacket'
	actions and number of 'DeliverPacket' actions can at most be 
	equal to the number of racks in the system.
	\end{itemize}
\end{enumerate}
%End of translations.