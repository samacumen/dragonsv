%Lists all the translated global requirements
% This must cover all possible scenario for each of the global requirements that we decided.
% I have tried to list out all possiblities (in a way that modal-u-calculus) can be written for these translations (sounding negative is important to formulate modal formula much easily). Refer to the report from last year. Feel free to include anything that I may have missed out. (Leave a comment also so that we are aware what changed)

This section lists the requirements from section \ref{sec:global_req} in terms of interactions described in section \ref{sec:ext_interactions}.

\begin{enumerate}
\item \textit{Each elevator, rack and conveyor belt contains at most one packet}
	\begin{itemize}
	\item There is a packet on the conveyor belt, so no
	packet is allowed to enter.
	\item There is no packet on the conveyor belt, so no 
	packet is allowed to leave.	
	\item There is a packet on the elevator, so no packet is 
	allowed to be loaded.
	\item There is no packet on the elevator, so no packet is 
	allowed to be unloaded.
	\item There is a packet on the rack, so no packet is 
	allowed to be stored.
	\item There is no packet on the rack, so no packet is 
	allowed to leave.
	\end{itemize}
\item \textit{ Packet is exchanged only when the elevator platform is at the same level as that of a conveyor belt}
	\begin{itemize}
	\item  If elevator platform is not on the same level of input 
	conveyor belt, no packet is loaded onto the elevator.
	\item If elevator platform is not the same level as the output 
	conveyor belt, no packet is loaded onto the conveyor belt.
	\end{itemize}

\item \textit{Packet is exchanged only when elevator platform is at the same level as that of a rack}
	\begin{itemize}
	\item  If position of rack and elevator platform are different, no
	packet can be stored onto the rack from the elevator.
	\item  If position of rack and elevator platform are different, no
	packet is loaded onto the elevator platform from the rack.
	\end{itemize}

\item \textit{The two elevators cannot be at the same position}
	\begin{itemize}
	\item  It cannot happen that two elevators are at the same 
	position.
	\end{itemize}	
	
\item \textit{The lower elevator must never pass the upper one}
	\begin{itemize}
	\item The lower elevator is always below the upper elevator.
	\item The lower elevator cannot go to the highest position.
	\item The upper elevator cannot go to the lowest position.
	\end{itemize}
	
\item \textit{Packets are always delivered in the same order as
	requested}	
	\begin{itemize}
	\item Reception of latest requested packet cannot happen before
	the reception of the previously 'unprocessed' requested packet.
	\end{itemize}
	
\item \textit{If a packet is ready to enter and there is a free
	position at the rack(s), it will be eventually accepted}
	\begin{itemize}
	\item If there is no free position in racks, packet must not be 
	accepted.
	\item If there is a free position in the racks and the packet is 
	ready to enter, it must be eventually stored.
	\end{itemize}
	
\item \textit{If a requested packet is in the system, it will be
	eventually delivered}
	\begin{itemize}
	\item If a packet is requested and it is in the system, it must be
	delivered eventually.
	\item If a packet is requested and it is not in the system, no 
	packet is delivered.
	\end{itemize}
	
\item \textit{If a packet is unable to be located, a unique alarm must 
	be generated}
	\begin{itemize}	
	\item If the requested packet is not located in the racks, a alarm
	is generated.
	\end{itemize}
		
\item \textit{The number of packets in the system can at most be equal to the number of racks}
	\begin{itemize}
	\item The number packets in the racks cannot be greater than the 
	number of racks in the system.
	\end{itemize}
\end{enumerate}
%End of translations.